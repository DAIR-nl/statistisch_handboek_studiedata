\documentclass[]{article}
\usepackage{lmodern}
\usepackage{amssymb,amsmath}
\usepackage{ifxetex,ifluatex}
\usepackage{fixltx2e} % provides \textsubscript
\ifnum 0\ifxetex 1\fi\ifluatex 1\fi=0 % if pdftex
  \usepackage[T1]{fontenc}
  \usepackage[utf8]{inputenc}
\else % if luatex or xelatex
  \ifxetex
    \usepackage{mathspec}
  \else
    \usepackage{fontspec}
  \fi
  \defaultfontfeatures{Ligatures=TeX,Scale=MatchLowercase}
\fi
% use upquote if available, for straight quotes in verbatim environments
\IfFileExists{upquote.sty}{\usepackage{upquote}}{}
% use microtype if available
\IfFileExists{microtype.sty}{%
\usepackage{microtype}
\UseMicrotypeSet[protrusion]{basicmath} % disable protrusion for tt fonts
}{}
\usepackage[margin=1in]{geometry}
\usepackage{hyperref}
\hypersetup{unicode=true,
            pdftitle={One sample T-toets},
            pdfauthor={Irene van der Staaij \& Megiel Kerkhoven},
            pdfborder={0 0 0},
            breaklinks=true}
\urlstyle{same}  % don't use monospace font for urls
\usepackage{color}
\usepackage{fancyvrb}
\newcommand{\VerbBar}{|}
\newcommand{\VERB}{\Verb[commandchars=\\\{\}]}
\DefineVerbatimEnvironment{Highlighting}{Verbatim}{commandchars=\\\{\}}
% Add ',fontsize=\small' for more characters per line
\usepackage{framed}
\definecolor{shadecolor}{RGB}{248,248,248}
\newenvironment{Shaded}{\begin{snugshade}}{\end{snugshade}}
\newcommand{\AlertTok}[1]{\textcolor[rgb]{0.94,0.16,0.16}{#1}}
\newcommand{\AnnotationTok}[1]{\textcolor[rgb]{0.56,0.35,0.01}{\textbf{\textit{#1}}}}
\newcommand{\AttributeTok}[1]{\textcolor[rgb]{0.77,0.63,0.00}{#1}}
\newcommand{\BaseNTok}[1]{\textcolor[rgb]{0.00,0.00,0.81}{#1}}
\newcommand{\BuiltInTok}[1]{#1}
\newcommand{\CharTok}[1]{\textcolor[rgb]{0.31,0.60,0.02}{#1}}
\newcommand{\CommentTok}[1]{\textcolor[rgb]{0.56,0.35,0.01}{\textit{#1}}}
\newcommand{\CommentVarTok}[1]{\textcolor[rgb]{0.56,0.35,0.01}{\textbf{\textit{#1}}}}
\newcommand{\ConstantTok}[1]{\textcolor[rgb]{0.00,0.00,0.00}{#1}}
\newcommand{\ControlFlowTok}[1]{\textcolor[rgb]{0.13,0.29,0.53}{\textbf{#1}}}
\newcommand{\DataTypeTok}[1]{\textcolor[rgb]{0.13,0.29,0.53}{#1}}
\newcommand{\DecValTok}[1]{\textcolor[rgb]{0.00,0.00,0.81}{#1}}
\newcommand{\DocumentationTok}[1]{\textcolor[rgb]{0.56,0.35,0.01}{\textbf{\textit{#1}}}}
\newcommand{\ErrorTok}[1]{\textcolor[rgb]{0.64,0.00,0.00}{\textbf{#1}}}
\newcommand{\ExtensionTok}[1]{#1}
\newcommand{\FloatTok}[1]{\textcolor[rgb]{0.00,0.00,0.81}{#1}}
\newcommand{\FunctionTok}[1]{\textcolor[rgb]{0.00,0.00,0.00}{#1}}
\newcommand{\ImportTok}[1]{#1}
\newcommand{\InformationTok}[1]{\textcolor[rgb]{0.56,0.35,0.01}{\textbf{\textit{#1}}}}
\newcommand{\KeywordTok}[1]{\textcolor[rgb]{0.13,0.29,0.53}{\textbf{#1}}}
\newcommand{\NormalTok}[1]{#1}
\newcommand{\OperatorTok}[1]{\textcolor[rgb]{0.81,0.36,0.00}{\textbf{#1}}}
\newcommand{\OtherTok}[1]{\textcolor[rgb]{0.56,0.35,0.01}{#1}}
\newcommand{\PreprocessorTok}[1]{\textcolor[rgb]{0.56,0.35,0.01}{\textit{#1}}}
\newcommand{\RegionMarkerTok}[1]{#1}
\newcommand{\SpecialCharTok}[1]{\textcolor[rgb]{0.00,0.00,0.00}{#1}}
\newcommand{\SpecialStringTok}[1]{\textcolor[rgb]{0.31,0.60,0.02}{#1}}
\newcommand{\StringTok}[1]{\textcolor[rgb]{0.31,0.60,0.02}{#1}}
\newcommand{\VariableTok}[1]{\textcolor[rgb]{0.00,0.00,0.00}{#1}}
\newcommand{\VerbatimStringTok}[1]{\textcolor[rgb]{0.31,0.60,0.02}{#1}}
\newcommand{\WarningTok}[1]{\textcolor[rgb]{0.56,0.35,0.01}{\textbf{\textit{#1}}}}
\usepackage{graphicx,grffile}
\makeatletter
\def\maxwidth{\ifdim\Gin@nat@width>\linewidth\linewidth\else\Gin@nat@width\fi}
\def\maxheight{\ifdim\Gin@nat@height>\textheight\textheight\else\Gin@nat@height\fi}
\makeatother
% Scale images if necessary, so that they will not overflow the page
% margins by default, and it is still possible to overwrite the defaults
% using explicit options in \includegraphics[width, height, ...]{}
\setkeys{Gin}{width=\maxwidth,height=\maxheight,keepaspectratio}
\IfFileExists{parskip.sty}{%
\usepackage{parskip}
}{% else
\setlength{\parindent}{0pt}
\setlength{\parskip}{6pt plus 2pt minus 1pt}
}
\setlength{\emergencystretch}{3em}  % prevent overfull lines
\providecommand{\tightlist}{%
  \setlength{\itemsep}{0pt}\setlength{\parskip}{0pt}}
\setcounter{secnumdepth}{0}
% Redefines (sub)paragraphs to behave more like sections
\ifx\paragraph\undefined\else
\let\oldparagraph\paragraph
\renewcommand{\paragraph}[1]{\oldparagraph{#1}\mbox{}}
\fi
\ifx\subparagraph\undefined\else
\let\oldsubparagraph\subparagraph
\renewcommand{\subparagraph}[1]{\oldsubparagraph{#1}\mbox{}}
\fi

%%% Use protect on footnotes to avoid problems with footnotes in titles
\let\rmarkdownfootnote\footnote%
\def\footnote{\protect\rmarkdownfootnote}

%%% Change title format to be more compact
\usepackage{titling}

% Create subtitle command for use in maketitle
\providecommand{\subtitle}[1]{
  \posttitle{
    \begin{center}\large#1\end{center}
    }
}

\setlength{\droptitle}{-2em}

  \title{One sample T-toets}
    \pretitle{\vspace{\droptitle}\centering\huge}
  \posttitle{\par}
    \author{Irene van der Staaij \& Megiel Kerkhoven}
    \preauthor{\centering\large\emph}
  \postauthor{\par}
      \predate{\centering\large\emph}
  \postdate{\par}
    \date{7/26/2019}


\begin{document}
\maketitle

\hypertarget{gebruik}{%
\subsection{Gebruik}\label{gebruik}}

Gebruik de \emph{One sample t-toets} bij een hypothese over het
gemiddelde in één groep, bijvoorbeeld gemiddeld
tentamencijfer.\footnote{\href{https://wikistatistiek.amc.nl/index.php/T-toets\#one_sample_t-toets}{Wiki
  Statistiek Academischi Medisch Centrum}}.

\hypertarget{voorbeeld-uit-het-onderwijs}{%
\subsection{Voorbeeld uit het
onderwijs}\label{voorbeeld-uit-het-onderwijs}}

De opleidingsdirecteur van de opleiding Economie wil weten of het
gemiddelde eindexamencijfer voor Wiskunde A van haar VWO studenten lager
is dan het landelijke gemiddelde, zodat zij kan bepalen of het zinnig is
om een zomercursus statistiek te starten.

H\textsubscript{0}: Het gemiddelde cijfer voor wiskunde A van VWO
studenten bij de opleiding Economie is een 6,9, µ = 6,9.

H\textsubscript{A}: Het gemiddelde cijfer voor wiskunde A van VWO
studenten bij de opleiding Economie is anders dan een 6,9, µ ≠ 6,9.

\hypertarget{assumpties}{%
\subsection{Assumpties}\label{assumpties}}

\hypertarget{normaliteit}{%
\subsubsection{Normaliteit}\label{normaliteit}}

De t-toets gaat er vanuit dat de data normaal verdeeld is, dit hoeft
vooral bij kleine steekproeven niet het geval te zijn. Checkt de data
visueel op normaliteit door het maken van een dataplot: een histogram of
een boxplot. Of een toets normaal verdeeld is wordt getoetst door de The
Kolmogorov-Smirnov Test en voor een kleinere steekproef de The
Shapiro-Wilk Test \footnote{\href{https://wiki.uva.nl/methodologiewinkel/index.php/Normaliteit}{UvA
  Wiki Methodologiewinkel}}. Wanneer aan de assumptie niet wordt
voldaan, voer de non-parametrische toets de Wilcoxon Signed Rank Test
uit. \footnote{\url{http://r-statistics.co/Statistical-Tests-in-R.html}}

\hypertarget{uitvoering-in-r}{%
\subsection{Uitvoering in R}\label{uitvoering-in-r}}

Voor de uitvoering in R is er een dummydataset genaamd Wiskunde\_cijfers
ingeladen met gemiddelde cijfers in Wiskunde A van 165 middelbare
scholieren.

\hypertarget{normaliteit-1}{%
\subsubsection{Normaliteit}\label{normaliteit-1}}

\begin{Shaded}
\begin{Highlighting}[]
\KeywordTok{hist}\NormalTok{(Wiskunde_cijfers)}
\end{Highlighting}
\end{Shaded}

\includegraphics{One_sample_T-toets_files/figure-latex/histogram voor normaliteit-1.pdf}

De histogram laat een belvorm zien vergelijkbaar aan een
normaalverdeling, veel waardes liggen rondom het gemiddelde en weinig
bij de staarten van de verdeling.

\hypertarget{section}{%
\subsubsection{}\label{section}}

\hypertarget{kolmogorov-smirnov}{%
\paragraph{Kolmogorov-Smirnov}\label{kolmogorov-smirnov}}

\begin{Shaded}
\begin{Highlighting}[]
\NormalTok{x <-}\KeywordTok{rnorm}\NormalTok{(}\DecValTok{230}\NormalTok{)}
\KeywordTok{ks.test}\NormalTok{(Wiskunde_cijfers, x)}
\end{Highlighting}
\end{Shaded}

\begin{verbatim}
## 
##  Two-sample Kolmogorov-Smirnov test
## 
## data:  Wiskunde_cijfers and x
## D = 1, p-value < 2.2e-16
## alternative hypothesis: two-sided
\end{verbatim}

De kolmogorov-smirnov test is een non-parametrische toets, die het
verschil in vorm tussen twee verdelingen toetst. In dit geval toetst
deze test het verschil tussen de normaal verdeling en de verdeling van
de steekproef.Wanneer de p-waarde kleiner is dan 0.05 is er een verschil
tussen beide verdelingen.

In dit geval is p \textgreater{} 0.05, dus is de verdeling van de data
niet significant verschillend van de normaal verdeling.

\hypertarget{shapiro-wilk-test}{%
\paragraph{Shapiro-Wilk Test}\label{shapiro-wilk-test}}

\begin{Shaded}
\begin{Highlighting}[]
\KeywordTok{shapiro.test}\NormalTok{(Wiskunde_cijfers)}
\end{Highlighting}
\end{Shaded}

\begin{verbatim}
## 
##  Shapiro-Wilk normality test
## 
## data:  Wiskunde_cijfers
## W = 0.96197, p-value = 0.0001752
\end{verbatim}

De Shapiro-Wilk Test is een soort gelijke test als de Kolmogorov-smirnov
test en wordt vooral gebruikt bij kleine steekproeven. Wanneer de
p-waarde groter is dan 0.05 is de verdeling van de data niet significant
verschillend van de normaal verdeling. Huidige p-waarde is 000, dus niet
significant verschillend.

\hypertarget{one-sample-t-toets}{%
\subsubsection{One sample T-toets}\label{one-sample-t-toets}}

\begin{Shaded}
\begin{Highlighting}[]
\KeywordTok{t.test}\NormalTok{(Wiskunde_cijfers, }\DataTypeTok{mu =} \FloatTok{6.9}\NormalTok{) }\CommentTok{# testen of het gemiddelde van onze studenten afwijkt van het landelijk gemiddelde (6.9)}
\end{Highlighting}
\end{Shaded}

\begin{verbatim}
## 
##  One Sample t-test
## 
## data:  Wiskunde_cijfers
## t = 1.2029, df = 164, p-value = 0.2307
## alternative hypothesis: true mean is not equal to 6.9
## 95 percent confidence interval:
##  6.835857 7.164143
## sample estimates:
## mean of x 
##         7
\end{verbatim}

\hypertarget{toelichting-op-de-output}{%
\subsection{Toelichting op de output}\label{toelichting-op-de-output}}

Het gemiddelde van de steekproef is ** Degrees of freedom (df) = n -1 =
50 -1 = 49 p-waarde: boven de 0.05, dus de H\textsubscript{0} wordt niet
verworpen. 95-betrouwbaarheidsinterval: Bij herhaling van het onderzoek
zal in 95\% van de gevallen een interval uitkomen waarbij mu er in valt.

\hypertarget{rapportage}{%
\subsection{Rapportage}\label{rapportage}}

\hypertarget{verder-lezen}{%
\subsection{Verder lezen}\label{verder-lezen}}

Open Intro Statistics Third Edition

\hypertarget{achtergrond-info}{%
\subsection{Achtergrond info}\label{achtergrond-info}}

De t-toets wordt ingezet wanneer de data numeriek en (gedeeltelijk)
normaal verdeeld is, maar de keuze welke t-toets men kiest is
afhankelijk van het aantal groepen wat je bekijkt en de relatie tussen
de twee groepen.


\end{document}
